\documentclass[12pt,letterpaper]{article}
\usepackage{fullpage}
\usepackage[top=2cm, bottom=4.5cm, left=2.5cm, right=2.5cm]{geometry}
\usepackage{amsmath,amsthm,amsfonts,amssymb,amscd}
\usepackage{lastpage}
\usepackage{enumerate}
\usepackage{fancyhdr}
\usepackage{mathrsfs}
\usepackage{xcolor}
\usepackage{graphicx}
\usepackage{listings}
\usepackage{hyperref}
\usepackage{tabularx}

\hypersetup{
  colorlinks=true,
  linkcolor=blue,
  linkbordercolor={0 0 1}
}
 
\definecolor{javared}{rgb}{0.6,0,0} % for strings
\definecolor{javagreen}{rgb}{0.25,0.5,0.35} % comments
\definecolor{javapurple}{rgb}{0.5,0,0.35} % keywords
\definecolor{javadocblue}{rgb}{0.25,0.35,0.75} % javadoc

\lstset{
    language=Java,
    basicstyle=\ttfamily,
    keywordstyle=\color{javapurple}\bfseries,
    stringstyle=\color{javared},
    commentstyle=\color{javagreen},
    morecomment=[s][\color{javadocblue}]{/**}{*/},
    numbers=left,
    numberstyle=\tiny\color{black},
    stepnumber=1,
    numbersep=10pt,
    tabsize=4,
    showspaces=false,
    showstringspaces=false
}

\renewcommand{\lstlistingname}{Snippet}

\setlength{\parindent}{0.0in}
\setlength{\parskip}{0.05in}

% Edit these as appropriate
\newcommand\course{ASRR Tech Training}
\newcommand\hwnumber{9}

\pagestyle{fancyplain}
\headheight 35pt
 
\chead{\textbf{\Large Module \hwnumber}}
\rhead{\course}
\lfoot{}
\cfoot{}
\rfoot{\small\thepage}
\headsep 1.5em

\begin{document}

\section*{\hwnumber. OOP: inheritance \&  polymorphism}
Resources:
\begin{description}
    \item \href{http://tutorials.jenkov.com/java/inheritance.html}{OOP: inheritance}
    \item \href{https://www.w3schools.com/java/java_polymorphism.asp}{OOP: polymorphism}
\end{description}


Problems
\begin{enumerate}
\item
Create a class named "Animal". It should have the following fields: name, weight and age. Also create the
constructor, getters and setters.
\begin{enumerate}
\item
Create a class named "Snake". It should inherit from class "Animal" and should have an extra field called
"venomous". Also create the constructor (the constructor should also initialize the fields from the superclass), getters and setters. Create a method called "slither". It should print that the snake is moving.

\item
Create a class named "Cat". It should inherit from class "Animal" and should have an extra field called
"furColor". Also create the constructor (the constructor should also initialize the fields from the superclass), getters and setters. Create a method called "meow". It should print "meow".

\item
Override the method getName from class Animal in class Cat. The method should return "Cat: " + the original name returned by the getName of the Animal class. So if a cat has a name "Bob" the method Cat\#getName should return "Cat: Bob".
\\ \\
\textbf{The following code should be written in your main method.}
\item
Create an array of 4 animals. This array should be initialized with both snakes AND cats. Iterate over the array and print the name of every animal.

\item
Iterate over the array of animals. Check whether the animal is a snake or a cat and call the appropriate meow or slither method.

\end{enumerate}

\newpage

\item
What does the code in the following snippet do? And why does it lead to this behavior?
\begin{lstlisting}
public class Entity {

    private String name;

    public Entity(String name) {
        this.name = name;
    }

    public String getName() {
        return name;
    }

}

public class Monster extends Entity {

    private String name;

    public Monster(String name) {
        super(name);
    }

    public void setName(String name) {
        this.name = name;
    }

}

public class Main {

    public static void main(String[] args) {
        Monster monster = new Monster("bob");
        monster.setName("ross");
        System.out.println(monster.getName());
    }

}
\end{lstlisting}

\item
What are the benefits of inheritance?

\item
What is polymorphism?

\end{enumerate}

\end{document}
