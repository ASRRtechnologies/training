\documentclass[12pt,letterpaper]{article}
\usepackage{fullpage}
\usepackage[top=2cm, bottom=4.5cm, left=2.5cm, right=2.5cm]{geometry}
\usepackage{amsmath,amsthm,amsfonts,amssymb,amscd}
\usepackage{lastpage}
\usepackage{enumerate}
\usepackage{fancyhdr}
\usepackage{mathrsfs}
\usepackage{xcolor}
\usepackage{graphicx}
\usepackage{listings}
\usepackage{hyperref}

\hypersetup{
  colorlinks=true,
  linkcolor=blue,
  linkbordercolor={0 0 1}
}
 
\definecolor{javared}{rgb}{0.6,0,0} % for strings
\definecolor{javagreen}{rgb}{0.25,0.5,0.35} % comments
\definecolor{javapurple}{rgb}{0.5,0,0.35} % keywords
\definecolor{javadocblue}{rgb}{0.25,0.35,0.75} % javadoc

\lstset{
    language=Java,
    basicstyle=\ttfamily,
    keywordstyle=\color{javapurple}\bfseries,
    stringstyle=\color{javared},
    commentstyle=\color{javagreen},
    morecomment=[s][\color{javadocblue}]{/**}{*/},
    numbers=left,
    numberstyle=\tiny\color{black},
    stepnumber=1,
    numbersep=10pt,
    tabsize=4,
    showspaces=false,
    showstringspaces=false
}

\renewcommand{\lstlistingname}{Snippet}

\setlength{\parindent}{0.0in}
\setlength{\parskip}{0.05in}

% Edit these as appropriate
\newcommand\course{ASRR Tech Training}
\newcommand\hwnumber{1}

\pagestyle{fancyplain}
\headheight 35pt
 
\chead{\textbf{\Large Module \hwnumber}}
\rhead{\course}
\lfoot{}
\cfoot{}
\rfoot{\small\thepage}
\headsep 1.5em

\begin{document}

\section*{\hwnumber. Methods}
Note: all your methods should be \lstinline{public static} in this module \\
Resources:
\begin{description}
    \item \href{https://www.geeksforgeeks.org/methods-in-java}{Methods}
    \item \href{https://www.geeksforgeeks.org/overloading-in-java/}{Method overloading}
    \item \href{https://stackoverflow.com/questions/9344305/what-is-short-circuiting-and-how-is-it-used-when-programming-in-java}{Short-circuit evaluation}
\end{description}


Problems
\begin{enumerate}
\item
Write a method that takes an angle in degrees as its input and returns the same angle in radians.

\item
Write a method that takes an integer n as its input and returns the factorial of n as its output.

\item
Write a method that takes two sides of a right triangle(as integers) as its input and returns the hypotenuse of the triangle as its output.

\item
Write a method that takes two booleans as its input. The method should mimic the behavior of the XOR logic gate.

\item
Write a method that takes three integers as its input and returns the smallest integer as its output.

\item
Write a method named "sum" that takes two integers as its input and returns the sum as its output.
Write another method named "sum" that takes three integers as its input and returns the sum as its output.
Explain why the compiler did not throw an error.

\item
Short-circuit evaluation is used to efficiently evaluate boolean expressions.
Write a program that visualizes(with print statements) the difference between the \lstinline{&&} and the \lstinline{&} operator on booleans.

\end{enumerate}

\end{document}
