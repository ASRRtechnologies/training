\documentclass[12pt,letterpaper]{article}
\usepackage{fullpage}
\usepackage[top=2cm, bottom=4.5cm, left=2.5cm, right=2.5cm]{geometry}
\usepackage{amsmath,amsthm,amsfonts,amssymb,amscd}
\usepackage{lastpage}
\usepackage{enumerate}
\usepackage{fancyhdr}
\usepackage{mathrsfs}
\usepackage{xcolor}
\usepackage{graphicx}
\usepackage{listings}
\usepackage{hyperref}
\usepackage{tabularx}

\hypersetup{
  colorlinks=true,
  linkcolor=blue,
  linkbordercolor={0 0 1}
}
 
\definecolor{javared}{rgb}{0.6,0,0} % for strings
\definecolor{javagreen}{rgb}{0.25,0.5,0.35} % comments
\definecolor{javapurple}{rgb}{0.5,0,0.35} % keywords
\definecolor{javadocblue}{rgb}{0.25,0.35,0.75} % javadoc

\lstset{
    language=Java,
    basicstyle=\ttfamily,
    keywordstyle=\color{javapurple}\bfseries,
    stringstyle=\color{javared},
    commentstyle=\color{javagreen},
    morecomment=[s][\color{javadocblue}]{/**}{*/},
    numbers=left,
    numberstyle=\tiny\color{black},
    stepnumber=1,
    numbersep=10pt,
    tabsize=4,
    showspaces=false,
    showstringspaces=false
}

\renewcommand{\lstlistingname}{Snippet}

\setlength{\parindent}{0.0in}
\setlength{\parskip}{0.05in}

% Edit these as appropriate
\newcommand\course{ASRR Tech Training}
\newcommand\hwnumber{5}

\pagestyle{fancyplain}
\headheight 35pt
 
\chead{\textbf{\Large Module \hwnumber}}
\rhead{\course}
\lfoot{}
\cfoot{}
\rfoot{\small\thepage}
\headsep 1.5em

\begin{document}

\section*{\hwnumber. Classes: introduction and attributes}
Note: attributes (fields) should NOT be \lstinline{static} in this module \\
Resources:
\begin{description}
    \item \href{https://www.w3schools.com/java/java_classes.asp}{Classes: introduction}
    \item \href{https://www.w3schools.com/java/java_classes.asp}{Classes: attributes}
\end{description}


Problems
\begin{enumerate}
\item
Create a class named "Pizza". It should have the following attributes: price, name and description. 
\begin{enumerate}
\item 
Write a method
that creates an instance of the class "Pizza". The method should initialize the object with appropriate values (e.g. \$9.50, "Margherita", "A very nice pizza").
\item
Write a \lstinline[]$static$ method that prints every attribute of a pizza instance.
\end{enumerate}

\item
Create a class named "Person". It should have the following attributes: age, name and friend. The attribute friend
could be another "Person"".

\begin{enumerate}
\item 
Write a method that creates an instance of class "Person".
The method should initialize the object with appropriate values (e.g. 33, "Bob").
The friend attribute of this instance should be initialized to another instance of class "Person" 
(the friend attribute of this person should be \lstinline[]$null$).
\item
Write a \lstinline[]$static$ method that prints every attribute of a person instance. It should print the friend attribute
RECURSIVELY.
\end{enumerate}


\end{enumerate}

\end{document}
