\documentclass[12pt,letterpaper]{article}
\usepackage{fullpage}
\usepackage[top=2cm, bottom=4.5cm, left=2.5cm, right=2.5cm]{geometry}
\usepackage{amsmath,amsthm,amsfonts,amssymb,amscd}
\usepackage{lastpage}
\usepackage{enumerate}
\usepackage{fancyhdr}
\usepackage{mathrsfs}
\usepackage{xcolor}
\usepackage{graphicx}
\usepackage{listings}
\usepackage{hyperref}
\usepackage{tabularx}

\hypersetup{
  colorlinks=true,
  linkcolor=blue,
  linkbordercolor={0 0 1}
}
 
\definecolor{javared}{rgb}{0.6,0,0} % for strings
\definecolor{javagreen}{rgb}{0.25,0.5,0.35} % comments
\definecolor{javapurple}{rgb}{0.5,0,0.35} % keywords
\definecolor{javadocblue}{rgb}{0.25,0.35,0.75} % javadoc

\lstset{
    language=Java,
    basicstyle=\ttfamily,
    keywordstyle=\color{javapurple}\bfseries,
    stringstyle=\color{javared},
    commentstyle=\color{javagreen},
    morecomment=[s][\color{javadocblue}]{/**}{*/},
    numbers=left,
    numberstyle=\tiny\color{black},
    stepnumber=1,
    numbersep=10pt,
    tabsize=4,
    showspaces=false,
    showstringspaces=false
}

\renewcommand{\lstlistingname}{Snippet}

\setlength{\parindent}{0.0in}
\setlength{\parskip}{0.05in}

% Edit these as appropriate
\newcommand\course{ASRR Tech Training}
\newcommand\hwnumber{8}

\pagestyle{fancyplain}
\headheight 35pt
 
\chead{\textbf{\Large Module \hwnumber}}
\rhead{\course}
\lfoot{}
\cfoot{}
\rfoot{\small\thepage}
\headsep 1.5em

\begin{document}

\section*{\hwnumber. OOP: encapsulation}
Note: try to make every field \lstinline[]$private$ \\
Resources:
\begin{description}
    \item \href{https://www.w3schools.com/java/java_modifiers.asp}{OOP: access modifiers}
    \item \href{https://beginnersbook.com/2013/05/encapsulation-in-java/}{OOP: encapsulation}
\end{description}


Problems
\begin{enumerate}
\item
Create a class named "Circle" It should have the following attributes: x, y and radius. Make sure that the attributes are all
\lstinline[]$private$.
\begin{enumerate}
\item
Create a constructor with parameters x, y and radius that initializes the attributes in this class.
\item
Create getters and setters for the every attribute.
\item
Create a method that determines whether a circle intersects another circle.
\begin{description}
    \item Example behavior:
    \item   
     \begin{tabular}{@{}rl@{}}
      Input: & a circle \\
      Output: & \lstinline[]$true$ if and only if the this circle intersects the other circle
     \end{tabular}
\end{description}
\end{enumerate}

\newpage

\item
What does the code in the following snippet do? And why does it lead to this behavior?
\begin{lstlisting}
public class Person {

    private int age = 0;

}

public class Main {

    public static void main(String[] args) {
        Person p = new Person();
        p.age++;
    }

}
\end{lstlisting}

\newpage


\item
What does the code in the following snippet do? And why does it lead to this behavior?
\begin{lstlisting}
public class Person {

    public int age;

    public Person(int age) {
        this.age = age;
    }

    private int getAge() {
        return age;
    }
}
    
public class Main {

    public static void main(String[] args) {
        Person person = new Person(11);
        System.out.println(person.age);
        System.out.println(person.getAge());
    }

}
\end{lstlisting}

\newpage

\item
What does the code in the following snippet do? And why does it lead to this behavior?
\begin{lstlisting}
    public class Person {

        private int age;

        public Person(int age) {
            this.age = age;
        }

        public int getAge() {
            return age;
        }

        private void setAge(int age) {
            this.age = age;
        }
    }

    public class Main {

        public static void main(String[] args) {
            Person person = new Person(11);
            System.out.println(person.getAge());
            person.setAge(1);
            System.out.println(person.getAge());
        }

    }
\end{lstlisting}

\item
What is the purpose of encapsulation?

\end{enumerate}

\end{document}
