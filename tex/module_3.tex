\documentclass[12pt,letterpaper]{article}
\usepackage{fullpage}
\usepackage[top=2cm, bottom=4.5cm, left=2.5cm, right=2.5cm]{geometry}
\usepackage{amsmath,amsthm,amsfonts,amssymb,amscd}
\usepackage{lastpage}
\usepackage{enumerate}
\usepackage{fancyhdr}
\usepackage{mathrsfs}
\usepackage{xcolor}
\usepackage{graphicx}
\usepackage{listings}
\usepackage{hyperref}
\usepackage{tabularx}

\hypersetup{
  colorlinks=true,
  linkcolor=blue,
  linkbordercolor={0 0 1}
}
 
\definecolor{javared}{rgb}{0.6,0,0} % for strings
\definecolor{javagreen}{rgb}{0.25,0.5,0.35} % comments
\definecolor{javapurple}{rgb}{0.5,0,0.35} % keywords
\definecolor{javadocblue}{rgb}{0.25,0.35,0.75} % javadoc

\lstset{
    language=Java,
    basicstyle=\ttfamily,
    keywordstyle=\color{javapurple}\bfseries,
    stringstyle=\color{javared},
    commentstyle=\color{javagreen},
    morecomment=[s][\color{javadocblue}]{/**}{*/},
    numbers=left,
    numberstyle=\tiny\color{black},
    stepnumber=1,
    numbersep=10pt,
    tabsize=4,
    showspaces=false,
    showstringspaces=false
}

\renewcommand{\lstlistingname}{Snippet}

\setlength{\parindent}{0.0in}
\setlength{\parskip}{0.05in}

% Edit these as appropriate
\newcommand\course{ASRR Tech Training}
\newcommand\hwnumber{3}

\pagestyle{fancyplain}
\headheight 35pt
 
\chead{\textbf{\Large Module \hwnumber}}
\rhead{\course}
\lfoot{}
\cfoot{}
\rfoot{\small\thepage}
\headsep 1.5em

\begin{document}

\section*{\hwnumber. Arrays}
Note 1: all your methods should be \lstinline{public static} in this module \\
Note 2: try to use the enhanced for loop as much as possible. \\
Resources:
\begin{description}
    \item \href{ https://www.webucator.com/tutorial/learn-java/arrays.cfm}{Arrays}
\end{description}


Problems
\begin{enumerate}
\item
Write a method that takes an array of integers as its input. The method should print every integer in the array.

\item
Write a method that takes an array of integers as its input and returns the sum of every integer in the array.

\item
Write a method that calculates the mean of an array.
\begin{description}
    \item Example behavior:
    \item   
     \begin{tabular}{@{}rl@{}}
      Input: & an array of integers \\
      Output: & the mean of the array (might not be an integer)
     \end{tabular}
\end{description}

\item
Write a method that normalizes a vector.
\begin{description}
    \item Example behavior:
    \item   
     \begin{tabular}{@{}rl@{}}
      Input: & a vector represented as a double array (\lstinline[]$double[]$) \\
      Output: & the normalized vector represented as a double array (\lstinline[]$double[]$)
     \end{tabular}
\end{description}

\item
Write a method that calculates the sum of two vectors.
\begin{description}
    \item Example behavior:
    \item   
     \begin{tabular}{@{}rl@{}}
      Input: & \lstinline[]$double[] a, double[] b$ (two vectors) \\
      Output: & the sum of the two vectors (\lstinline[]$double[]$)
     \end{tabular}
\end{description}

\item
Write a method that replaces every 1 in an integer array with a 0.
\begin{description}
    \item Example behavior:
    \item   
     \begin{tabular}{@{}rl@{}}
      Input: & \lstinline[]$int[]$ \\
      Output: & the same integer array, but every 1 should be replaced with a 0
     \end{tabular}
\end{description}

\item
Write a method that removes every 1 in an integer array.
\begin{description}
    \item Example behavior:
    \item   
     \begin{tabularx}{\linewidth}{ r X }
      Input: & \lstinline[]$int[]$ \\
      Output: & the same integer array, but every 1 should be removed from the array 
      (thus the output array may be shorter than the input array)
     \end{tabularx}
\end{description}

\item
Write a method that concatenates two arrays.
\begin{description}
    \item Example behavior:
    \item   
     \begin{tabularx}{\linewidth}{ r X }
      Input: & \lstinline[]$int[]a, int[] b$ \\
      Output: & all the elements from array a and array b in a single array (\lstinline[]$int[]$)
     \end{tabularx}
\end{description}

\item
Write a method that removes a section of an array.
\begin{description}
    \item Example behavior:
    \item   
     \begin{tabularx}{\linewidth}{ r X }
      Input: & \lstinline[]$int[]a, int startIndex, int endIndex$ \\
      Output: & array a, but every element from \lstinline[]$startIndex$ till \lstinline[]$endIndex$ should be removed
      (both \lstinline[]$startIndex$ and \lstinline[]$endIndex$ are inclusive)
     \end{tabularx}
\end{description}

\item
Write the following method that sorts an array (do not use any sorting library, instead try to write the algorithm yourself)
\begin{description}
    \item Example behavior:
    \item   
     \begin{tabularx}{\linewidth}{ r X }
      Input: & \lstinline[]$int[] a$ \\
      Output: & the sorted version of array a (the order does not matter)
     \end{tabularx}
\end{description}

\end{enumerate}

\end{document}
