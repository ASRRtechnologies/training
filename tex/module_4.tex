\documentclass[12pt,letterpaper]{article}
\usepackage{fullpage}
\usepackage[top=2cm, bottom=4.5cm, left=2.5cm, right=2.5cm]{geometry}
\usepackage{amsmath,amsthm,amsfonts,amssymb,amscd}
\usepackage{lastpage}
\usepackage{enumerate}
\usepackage{fancyhdr}
\usepackage{mathrsfs}
\usepackage{xcolor}
\usepackage{graphicx}
\usepackage{listings}
\usepackage{hyperref}
\usepackage{tabularx}

\hypersetup{
  colorlinks=true,
  linkcolor=blue,
  linkbordercolor={0 0 1}
}
 
\definecolor{javared}{rgb}{0.6,0,0} % for strings
\definecolor{javagreen}{rgb}{0.25,0.5,0.35} % comments
\definecolor{javapurple}{rgb}{0.5,0,0.35} % keywords
\definecolor{javadocblue}{rgb}{0.25,0.35,0.75} % javadoc

\lstset{
    language=Java,
    basicstyle=\ttfamily,
    keywordstyle=\color{javapurple}\bfseries,
    stringstyle=\color{javared},
    commentstyle=\color{javagreen},
    morecomment=[s][\color{javadocblue}]{/**}{*/},
    numbers=left,
    numberstyle=\tiny\color{black},
    stepnumber=1,
    numbersep=10pt,
    tabsize=4,
    showspaces=false,
    showstringspaces=false
}

\renewcommand{\lstlistingname}{Snippet}

\setlength{\parindent}{0.0in}
\setlength{\parskip}{0.05in}

% Edit these as appropriate
\newcommand\course{ASRR Tech Training}
\newcommand\hwnumber{4}

\pagestyle{fancyplain}
\headheight 35pt
 
\chead{\textbf{\Large Module \hwnumber}}
\rhead{\course}
\lfoot{}
\cfoot{}
\rfoot{\small\thepage}
\headsep 1.5em

\begin{document}

\section*{\hwnumber. n-dimensional arrays}
Note: all your methods should be \lstinline{public static} in this module \\
Resources:
\begin{description}
    \item \href{https://www.programiz.com/java-programming/multidimensional-array}{n-dimensional arrays}
\end{description}


Problems
\begin{enumerate}
\item
Write a method that takes a two-dimensional array of integers as its input. The method should print every integer in the array.

\item
Write a method that takes a two-dimensional array of integers as its input. The method should print every integer in every row with an odd index.
\begin{description}
    \item Example behavior:
    \item   
     \begin{tabularx}{\linewidth}{ r X }
      Input: & \lstinline[]$double[][] a = \{ \{1, 2\}, \{3, 4\}, \{6, 7\}\}$ \\
      Output: & \lstinline[]$3, 4$
     \end{tabularx}
\end{description}

\item
Write a method that takes a three-dimensional array of integers as its input and returns the sum of every integer in the array.

\item
Write a method that takes two-dimensional array of integers as its input. The method should determine whether the
two-dimensional array is shaped like a square.
\begin{description}
    \item Example behavior:
    \item   
     \begin{tabularx}{\linewidth}{ r X }
      Input: & \lstinline[]$double[][] a = \{ \{1.1, 2\}, \{3, 4.5\}\}$ \\
      Output: & \lstinline[]$true$
     \end{tabularx}
    \item   
     \begin{tabularx}{\linewidth}{ r X }
      Input: & \lstinline[]$double[][] a = \{ \{1.1, 2\}, \{3\}\}$ \\
      Output: & \lstinline[]$false$
     \end{tabularx}
\end{description}

\item
Write a method that multiplies a matrix by a scalar.
\begin{description}
    \item Example behavior:
    \item   
     \begin{tabularx}{\linewidth}{ r X }
      Input: & \lstinline[]$double scalar, double[][] matrix$ \\
      Output: & the calculated matrix \lstinline[]$double[][]$
     \end{tabularx}
\end{description}

\item
Write a method that multiplies a matrix by a vector. You should return an empty vector when the input is invalid.
\begin{description}
    \item Example behavior:
    \item   
     \begin{tabularx}{\linewidth}{ r X }
      Input: & \lstinline[]$double[] vector, double[][] matrix$ \\
      Output: & the calculated vector \lstinline[]$double[]$
     \end{tabularx}
\end{description}

\item
Write a method that calculates the sum of two matrices. You should return an empty matrix when the input is invalid.
\begin{description}
    \item Example behavior:
    \item   
     \begin{tabularx}{\linewidth}{ r X }
      Input: & \lstinline[]$double[][] a, double[][] b$ \\
      Output: & the sum of matrix \lstinline[]$a$ and matrix \lstinline[]$b$ as \lstinline[]$double[][]$
     \end{tabularx}
\end{description}

\end{enumerate}

\end{document}
